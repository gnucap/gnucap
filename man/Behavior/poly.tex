%$Id: poly.tex,v 20.15 2001/10/30 03:59:06 al Exp $
% man behavior poly .
% Copyright (C) 2001 Albert Davis
% Author: Albert Davis <aldavis@ieee.org>
%
% This file is part of "GnuCap", the Gnu Circuit Analysis Package
%
% This program is free software; you can redistribute it and/or modify
% it under the terms of the GNU General Public License as published by
% the Free Software Foundation; either version 2, or (at your option)
% any later version.
%
% This program is distributed in the hope that it will be useful,
% but WITHOUT ANY WARRANTY; without even the implied warranty of
% MERCHANTABILITY or FITNESS FOR A PARTICULAR PURPOSE.  See the
% GNU General Public License for more details.
%
% You should have received a copy of the GNU General Public License
% along with this program; if not, write to the Free Software
% Foundation, Inc., 59 Temple Place - Suite 330, Boston, MA
% 02111-1307, USA.
%------------------------------------------------------------------------
\section{{\tt POLY}: Polynomial nonlinear transfer function}
%------------------------------------------------------------------------
\subsection{Syntax}
\begin{verse}
{\tt POLY} {\it c0 c1 c2 c3 ...}\\
{\tt POLY} {\it c0 c1 c2 c3 ... args}
\end{verse}
%------------------------------------------------------------------------
\subsection{Purpose}

Defines a transfer function by a one dimensional polynomial.
%------------------------------------------------------------------------
\subsection{Comments}

For capacitors, this function defines {\em charge} as a function of
voltage.  For inductors, it defines {\em flux} as a function of
current.  If you have the coefficients defining capacitance or
inductance, prepending a ``0'' to the list will turn it into the
correct form for GnuCap.

For fixed sources, it defines voltage or current as a polynomial
function of time.

The transfer function is defined by:

\begin{verbatim}
out = c0 + (c1*in) + (c2*in^2) + ....
\end{verbatim}
%------------------------------------------------------------------------
\subsection{Parameters}

\begin{description}

\item[{\tt MIN} = {\it x}] Minimum output value (clipping).
(Default = -infinity.)

\item[{\tt MAX} = {\it x}] Maximum output value (clipping).
(Default = infinity)

\item[{\tt ABS}] Absolute value, truth value.  (Default = false).  If
set to true, the result will be always positive.

\end{description}
%------------------------------------------------------------------------
%\subsection{Example} 
%------------------------------------------------------------------------
%------------------------------------------------------------------------
