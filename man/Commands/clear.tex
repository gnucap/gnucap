%$Id: clear.tex,v 25.95 2006/08/26 01:26:53 al Exp $ -*- LaTeX -*-
% man commands clear
% Copyright (C) 2001 Albert Davis
% Author: Albert Davis <aldavis@ieee.org>
%
% This file is part of "Gnucap", the Gnu Circuit Analysis Package
%
% This program is free software; you can redistribute it and/or modify
% it under the terms of the GNU General Public License as published by
% the Free Software Foundation; either version 2, or (at your option)
% any later version.
%
% This program is distributed in the hope that it will be useful,
% but WITHOUT ANY WARRANTY; without even the implied warranty of
% MERCHANTABILITY or FITNESS FOR A PARTICULAR PURPOSE.  See the
% GNU General Public License for more details.
%
% You should have received a copy of the GNU General Public License
% along with this program; if not, write to the Free Software
% Foundation, Inc., 51 Franklin Street, Fifth Floor, Boston, MA
% 02110-1301, USA.
%------------------------------------------------------------------------
\section{{\tt CLEAR} command}
\index{clear command}
\index{clear circuit}
\index{erase}
\index{title: clear}
%------------------------------------------------------------------------
\subsection{Syntax}
\begin{verse}
{\tt clear}
\end{verse}
%------------------------------------------------------------------------
\subsection{Purpose}

Deletes the entire circuit, and blanks the title.
%------------------------------------------------------------------------
\subsection{Comments}

The entire word {\tt clear} is required.

{\tt Clear} is similar to, but a little more drastic than {\tt delete all}.

After deleting anything, there is no way to get it back.

See also: {\tt delete} command.
%------------------------------------------------------------------------
\subsection{Examples}

\begin{description}

\item[{\tt clear}] Delete the entire circuit.

\end{description}
%------------------------------------------------------------------------
%------------------------------------------------------------------------
