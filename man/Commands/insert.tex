%$Id: insert.tex,v 21.13 2002/03/25 05:37:03 al Exp $
% man commands insert .
% Copyright (C) 2001 Albert Davis
% Author: Albert Davis <aldavis@ieee.org>
%
% This file is part of "Gnucap", the Gnu Circuit Analysis Package
%
% This program is free software; you can redistribute it and/or modify
% it under the terms of the GNU General Public License as published by
% the Free Software Foundation; either version 2, or (at your option)
% any later version.
%
% This program is distributed in the hope that it will be useful,
% but WITHOUT ANY WARRANTY; without even the implied warranty of
% MERCHANTABILITY or FITNESS FOR A PARTICULAR PURPOSE.  See the
% GNU General Public License for more details.
%
% You should have received a copy of the GNU General Public License
% along with this program; if not, write to the Free Software
% Foundation, Inc., 59 Temple Place - Suite 330, Boston, MA
% 02111-1307, USA.
%------------------------------------------------------------------------
\section{{\tt INSERT} command}
\index{insert command}
\index{nodes: insert}
%------------------------------------------------------------------------
\subsection{Syntax}
\begin{verse}
{\tt INsert} {\it node}\\
{\tt INsert} {\it node, count}
\end{verse}
%------------------------------------------------------------------------
\subsection{Purpose}

Open up node numbers inside a circuit.
%------------------------------------------------------------------------
\subsection{Comments}

To open up an internal node, enter {\tt insert} followed by the number and
how many.  All node numbers higher than the first number will be raised by
the second.  The second (how many) is optional.  If omitted, 1 will be
assumed.
%------------------------------------------------------------------------
\subsection{Examples}

\begin{description}

\item[{\tt insert 8 3}] Insert 3 nodes before node 8.  Adds 3 nodes (8,9,10)
with no connections.  Old node numbers 8 and higher have 3 added to them to
make room.  Old node 8 is now 11, 9 is now 12, 10 is now 13, 11 is 14, etc.

\item[{\tt insert 6}] Insert one node at 6.  Old nodes 6 and higher are
incremented by 1.  Old node 6 is now 7, 7 is 8, etc.

\end{description}
%------------------------------------------------------------------------
%------------------------------------------------------------------------
