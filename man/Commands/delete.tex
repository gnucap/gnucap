%$Id: delete.tex,v 21.13 2002/03/25 05:37:03 al Exp $
% man commands delete .
% Copyright (C) 2001 Albert Davis
% Author: Albert Davis <aldavis@ieee.org>
%
% This file is part of "Gnucap", the Gnu Circuit Analysis Package
%
% This program is free software; you can redistribute it and/or modify
% it under the terms of the GNU General Public License as published by
% the Free Software Foundation; either version 2, or (at your option)
% any later version.
%
% This program is distributed in the hope that it will be useful,
% but WITHOUT ANY WARRANTY; without even the implied warranty of
% MERCHANTABILITY or FITNESS FOR A PARTICULAR PURPOSE.  See the
% GNU General Public License for more details.
%
% You should have received a copy of the GNU General Public License
% along with this program; if not, write to the Free Software
% Foundation, Inc., 59 Temple Place - Suite 330, Boston, MA
% 02111-1307, USA.
%------------------------------------------------------------------------
\section{{\tt DELETE} command}
\index{delete command}
\index{clear circuit}
\index{erase}
\index{remove parts}
%------------------------------------------------------------------------
\subsection{Syntax}
\begin{verse}
{\tt DELete} {\it label} ...\\
{\tt DELete ALL}
\end{verse}
%------------------------------------------------------------------------
\subsection{Purpose}

Remove a line, or a group of lines, from the circuit description.
%------------------------------------------------------------------------
\subsection{Comments}

To delete a part, by label, enter the label.  (Example `DEL R15'.)
Wildcards `{\tt *}' and `{\tt ?}' are allowed, in which case, all that match
are deleted.

To delete the entire circuit, the entire word {\tt ALL} must be entered.
(Example `DEL ALL'.)

After deleting anything, there is usually no way to get it back, but if a
fault had been applied (see {\tt fault} command) {\tt restore} may have
surprising results.
%------------------------------------------------------------------------
\subsection{Examples}

\begin{description}

\item[{\tt delete all}] Delete the entire circuit, but save the title.

\item[{\tt del R12}] Delete {\tt R12}.

\item[{\tt del R12 C3}] Delete {\tt R12} and {\tt C3}.

\item[{\tt del R*}] Delete all resistors.  (Also, any models and subcircuits
starting with {\tt R}.)

\end{description}
%------------------------------------------------------------------------
%------------------------------------------------------------------------
