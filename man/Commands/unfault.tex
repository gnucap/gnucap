%$Id: unfault.tex,v 25.95 2006/08/26 01:26:53 al Exp $ -*- LaTeX -*-
% man commands unfault .
% Copyright (C) 2001 Albert Davis
% Author: Albert Davis <aldavis@ieee.org>
%
% This file is part of "Gnucap", the Gnu Circuit Analysis Package
%
% This program is free software; you can redistribute it and/or modify
% it under the terms of the GNU General Public License as published by
% the Free Software Foundation; either version 2, or (at your option)
% any later version.
%
% This program is distributed in the hope that it will be useful,
% but WITHOUT ANY WARRANTY; without even the implied warranty of
% MERCHANTABILITY or FITNESS FOR A PARTICULAR PURPOSE.  See the
% GNU General Public License for more details.
%
% You should have received a copy of the GNU General Public License
% along with this program; if not, write to the Free Software
% Foundation, Inc., 51 Franklin Street, Fifth Floor, Boston, MA
% 02110-1301, USA.
%------------------------------------------------------------------------
\section{{\tt UNFAULT} command}
\index{faults: restore}
\index{unfault command}
\index{change values: temporary}
%------------------------------------------------------------------------
\subsection{Syntax}
\begin{verse}
{\tt unfault}
\end{verse}
%------------------------------------------------------------------------
\subsection{Purpose}

Undo any action from {\tt fault} commands.
%------------------------------------------------------------------------
\subsection{Comments}

This command reverses the action of all {\tt fault} commands.

It will also clean up any side effects of an aborted {\tt sweep} command.

{\tt Unfault} is automatically invoked on any {\tt clear} command.

If you change the circuit in any other way, {\tt unfault} will bring back the
old on top of the changes.  This can bring on some surprises.
%------------------------------------------------------------------------
\subsection{Example}

\begin{description}

\item[{\tt fault R66=1k}] R66 now has a value of 1k, regardless of what it
was before.

\item[{\tt unfault}] Clears all faults.  In this case, R66 has its old value
again.

\end{description}

{\tt unfault} can bring on surprises.  Consider this sequence ...

\begin{verbatim}
V1   1   0    ac  1
C3   1   2    1u
R4   2   0    10k
\end{verbatim}

\begin{description}

\item[{\tt fault C3=100p}] C3 is 100 picofarads, for now.

\item[{\tt modify C3=220p}] C3 is 220 pf, for now.  It will be restored.

\item[{\tt modify R4=1k}] R4 is 1k.  It will not be restored.

\item[{\tt restore}] C3 back to 1 uf, but R4 still 1k.

\end{description}
%------------------------------------------------------------------------
%------------------------------------------------------------------------
