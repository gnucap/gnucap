%$Id: i.tex,v 25.95 2006/08/26 01:26:53 al Exp $ -*- LaTeX -*-
% man circuit i .
% Copyright (C) 2001 Albert Davis
% Author: Albert Davis <aldavis@ieee.org>
%
% This file is part of "Gnucap", the Gnu Circuit Analysis Package
%
% This program is free software; you can redistribute it and/or modify
% it under the terms of the GNU General Public License as published by
% the Free Software Foundation; either version 2, or (at your option)
% any later version.
%
% This program is distributed in the hope that it will be useful,
% but WITHOUT ANY WARRANTY; without even the implied warranty of
% MERCHANTABILITY or FITNESS FOR A PARTICULAR PURPOSE.  See the
% GNU General Public License for more details.
%
% You should have received a copy of the GNU General Public License
% along with this program; if not, write to the Free Software
% Foundation, Inc., 51 Franklin Street, Fifth Floor, Boston, MA
% 02110-1301, USA.
%------------------------------------------------------------------------
\section{{\tt I}: Independent Current Source}
%------------------------------------------------------------------------
\subsection{Syntax}
\subsubsection{Device}
\begin{verse}
{\tt I}{\it xxxxxxx n+ n--} {\it value}\\
{\tt I}{\it xxxxxxx n+ n--} {\it expression}\\
{\tt I}{\it xxxxxxx n+ n--} {\tt GENERATOR} {\it scale}\\
{\tt .isource} {\it label n+ n-- expression}
\end{verse}
\subsubsection{Model (optional)}
\begin{verse}
{\tt .model} {\it mname} {\tt TABLE} \{{\it args}\}
\end{verse}
%------------------------------------------------------------------------
\subsection{Purpose}

Independent current source.
%------------------------------------------------------------------------
\subsection{Comments}

{\it N+} and {\it n--} are the positive and negative element nodes,
respectively.  Positive current flow is from the positive node, through the
source, to the negative node.  {\it Value} is the current in Amperes.

All of the SPICE time dependent functions ({\tt pulse}, {\tt sin}, {\tt
exp}, {\tt pwl}, and {\tt sffm} are supported.  An additional
function {\tt generator} emulates a laboratory type function generator, for
a more convenient signal input to the circuit.
See the chapter ``Behavioral modeling'' for more information.

You may specify the {\it value} in any of these forms:

\begin{enumerate}
  
\item
A simple value.  This is the DC current.
  
\item
An expression, as described in the behavioral modeling chapter.  The
expression can specify the current as a function of time.

\item
A {\it model}, as described in the behavioral modeling chapter.  The
{\tt table} model describes a table of current vs. time.

\item
The keyword {\tt generator} which specifies a time dependent waveform
set by the {\tt generator} command for transient analysis, or a swept
current for AC analysis.

\end{enumerate}
%------------------------------------------------------------------------
\subsection{Probes}

The standard probes for all basic elements are all available.  See the
{\tt print} command for documentation.
%------------------------------------------------------------------------
%------------------------------------------------------------------------
