%$Id: t.tex,v 25.95 2006/08/26 01:26:53 al Exp $ -*- LaTeX -*-
% man circuit t .
% Copyright (C) 2001 Albert Davis
% Author: Albert Davis <aldavis@ieee.org>
%
% This file is part of "Gnucap", the Gnu Circuit Analysis Package
%
% This program is free software; you can redistribute it and/or modify
% it under the terms of the GNU General Public License as published by
% the Free Software Foundation; either version 2, or (at your option)
% any later version.
%
% This program is distributed in the hope that it will be useful,
% but WITHOUT ANY WARRANTY; without even the implied warranty of
% MERCHANTABILITY or FITNESS FOR A PARTICULAR PURPOSE.  See the
% GNU General Public License for more details.
%
% You should have received a copy of the GNU General Public License
% along with this program; if not, write to the Free Software
% Foundation, Inc., 51 Franklin Street, Fifth Floor, Boston, MA
% 02110-1301, USA.
%------------------------------------------------------------------------
\section{{\tt T}: Transmission Line}
%------------------------------------------------------------------------
\subsection{Syntax}
\begin{verse}
{\tt T}{\it xxxxxxx n1+ n1-- n2+ n2--} \{{\it args}\}\\
{\tt .tline} {\it xxxxxxx n1+ n1-- n2+ n2--} \{{\it args}\}
\end{verse}
%------------------------------------------------------------------------
\subsection{Purpose}

Lossless transmission line.
%------------------------------------------------------------------------
\subsection{Comments}

{\it N1+} and {\it n1--} are the nodes at one end.  {\it N2+} and {\it
n2--} are the nodes at the other end.

The parameters {\tt TD}, {\tt Freq}, and {\tt NL} determine the length
of the line.  Either {\tt TD} or {\tt Freq} and {\tt NL} must be
specified.  If only {\tt Freq} is specified, {\tt NL} is assumed to be
.25.  The other will be calculated based on the one you specify.  If
you specify too many parameters, {\tt Freq} and {\tt NL} dominate, and
a warning is issued.
%------------------------------------------------------------------------
\subsection{Element Parameters}

Many parameters are offered.  You should not specify them all.

\subsubsection{Parameters that always work.}

\begin{description}

\item[{\tt LEN} = {\it x}]
Length multiplier.  (Default = 1) The effective length, regardless of
its method of calculation is multiplied by this number.

\end{description}


\subsubsection{Direct specification of electrical characteristics.}

\begin{description}

\item[{\tt Z0} = {\it x}]
Characteristic impedance.  If not specified, it is calculated by
$\sqrt(L/C)$.  If neither {\tt Z0} nor {\tt L} and {\tt C} are
specified, the default value is 50 Ohms.

\item[{\tt TD} = {\it x}] Time delay.  If not specified, it will be calculated, either by $NL/FREQ$ or by $\sqrt(L C)$.

\item[{\tt FREQ} = {\it x}] Frequency for NL.

\item[{\tt NL} = {\it x}] Number of wavelengths at {\it Freq}.

\end{description}

\subsubsection{Physical parameters}

\begin{description}

\item[{\tt L} = {\it x}]
Inductance per unit length.  This value is used only if {\tt Z0} and
{\tt TD} are not specified.

\item[{\tt C} = {\it x}]
Capacitance per unit length.  This value is used only if {\tt Z0} and
{\tt TD} are not specified.

\end{description}
%------------------------------------------------------------------------
\subsection{Probes}

The standard probes for all basic elements are all available.  See the
{\tt print} command for documentation.
%------------------------------------------------------------------------
%------------------------------------------------------------------------
