%$Id: f.tex,v 25.95 2006/08/26 01:26:53 al Exp $ -*- LaTeX -*-
% man circuit f .
% Copyright (C) 2001 Albert Davis
% Author: Albert Davis <aldavis@ieee.org>
%
% This file is part of "Gnucap", the Gnu Circuit Analysis Package
%
% This program is free software; you can redistribute it and/or modify
% it under the terms of the GNU General Public License as published by
% the Free Software Foundation; either version 2, or (at your option)
% any later version.
%
% This program is distributed in the hope that it will be useful,
% but WITHOUT ANY WARRANTY; without even the implied warranty of
% MERCHANTABILITY or FITNESS FOR A PARTICULAR PURPOSE.  See the
% GNU General Public License for more details.
%
% You should have received a copy of the GNU General Public License
% along with this program; if not, write to the Free Software
% Foundation, Inc., 51 Franklin Street, Fifth Floor, Boston, MA
% 02110-1301, USA.
%------------------------------------------------------------------------
\section{{\tt F}: Current Controlled Current Source}
%------------------------------------------------------------------------
\subsection{Syntax}
\subsubsection{Device}
\begin{verse}
{\tt F}{\it xxxxxxx n+ n-- ce} {\it value}\\
{\tt F}{\it xxxxxxx n+ n-- ce} {\it expression}\\
{\tt .CCCS} {\it label n+ n-- ce} {\it expression}
\end{verse}
\subsubsection{Model (optional)}
\begin{verse}
{\tt .model} {\it mname} {\tt TABLE} \{{\it args}\}
\end{verse}
%------------------------------------------------------------------------
\subsection{Purpose}

Current controlled current source, or current gain block.
%------------------------------------------------------------------------
\subsection{Comments}

{\it N+} and {\it n--} are the positive and negative element (output)
nodes, respectively.  Current flow is from the positive node,
through the source, to the negative node.  {\it Ce} is the name of
an element through which the controlling current flows.  The
direction of positive controlling current is from the positive
node, through the element, to the negative node of {\it ce}.  {\it
Value} is the current gain.

The controlling element can be any simple two terminal element.
Unlike SPICE, it does not need to be a voltage source.

You may specify the {\it value} in any of these forms:

\begin{enumerate}
  
\item
A simple value.  This is the current gain.
  
\item
An expression, as described in the behavioral modeling chapter.  The
expression can specify the output current as a function of input
current, or the current gain as a function of time.

\item
A {\it model}, as described in the behavioral modeling chapter.  The
{\tt table} model describes a table of output current vs. input
current.

\end{enumerate}
%------------------------------------------------------------------------
\subsection{Probes}

The standard probes for all basic elements are all available.  See the
{\tt print} command for documentation.
%------------------------------------------------------------------------
%------------------------------------------------------------------------
