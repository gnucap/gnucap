%$Id: c.tex,v 21.13 2002/03/25 05:37:03 al Exp $
% man circuit c .
% Copyright (C) 2001 Albert Davis
% Author: Albert Davis <aldavis@ieee.org>
%
% This file is part of "Gnucap", the Gnu Circuit Analysis Package
%
% This program is free software; you can redistribute it and/or modify
% it under the terms of the GNU General Public License as published by
% the Free Software Foundation; either version 2, or (at your option)
% any later version.
%
% This program is distributed in the hope that it will be useful,
% but WITHOUT ANY WARRANTY; without even the implied warranty of
% MERCHANTABILITY or FITNESS FOR A PARTICULAR PURPOSE.  See the
% GNU General Public License for more details.
%
% You should have received a copy of the GNU General Public License
% along with this program; if not, write to the Free Software
% Foundation, Inc., 59 Temple Place - Suite 330, Boston, MA
% 02111-1307, USA.
%------------------------------------------------------------------------
\section{{\tt C}: Capacitor}
%------------------------------------------------------------------------
\subsection{Syntax}
\begin{verse}
  {\tt C}{\it xxxxxxx n+ n-- value}\\
  {\tt C}{\it xxxxxxx n+ n-- expression}\\
  {\tt C}{\it xxxxxxx n+ n-- value} 
        \{{\tt IC=}{\it initial-voltage}\}\\
  {\tt C}{\it xxxxxxx n+ n-- model} \{{\tt L=}{\it length}\}
        \{{\tt W=}{\it width}\} \{{\tt TEMP=}{\it temperature}\}
        \{{\tt IC=}{\it initial-voltage}\}\\
  {\tt .CAPacitor} {\it label n+ n-- expression}
\end{verse}
%------------------------------------------------------------------------
\subsection{Purpose}

Capacitor, or general charge storage element.
%------------------------------------------------------------------------
\subsection{Probes}

The following probes (Transient, DC, and OP analysis) are available in
addition to those available for all basic elements.

\begin{description}
  
\item[{\tt DT}] Time step.  The internal time step used for this
  device for numerical integration.  It is not necessarily the same as
  the global time step.
  
\item[{\tt TIME}] Time.  The time of the most recent calculation of
  this device.  It is not necessarily the same as the global time.
  
\item[{\tt TIMEOLD}] The time of the previous calculation of this
  device.  It is not necessarily the same as the global time.
  
\item[{\tt TIMEFuture}] The latest recommended time for the next
  sample, as determined by this device.  The actual time will probably
  be sooner than this.
  
\item[{\tt CHarge}] The charge stored in this capacitor.
  
\item[{\tt Q}] The same as {\tt Charge}.
  
\item[{\tt Capacitance}] The effective capacitance of this device.
  For a fixed capacitor, it is constant.  It will vary if this device
  is nonlinear.
  
\item[{\tt DQDT}] The time derivative of charge.  Hopefully this is
  the same as current, but it is calculated a different way and can be
  used as an accuracy check.
  
\item[{\tt DQ}] The change in charge compared to the previous sample.
  Its primary use is in debugging models and numerical problems.

\end{description}
%------------------------------------------------------------------------
\subsection{Comments}

{\it N+} and {\it n--} are the positive and negative element nodes,
respectively.  {\it Value} is the capacitance in Farads.

The (optional) initial condition is the initial (time = 0) value of
the capacitor voltage (in Volts).  Note that the initial conditions
(if any) apply only if the {\tt UIC} option is specified on the {\tt
  transient} command.

You may specify the {\it value} in one of three forms:

\begin{enumerate}
  
\item A simple value.  This is the capacitance in Farads.
  
\item An expression, as described in the behavioral modeling chapter.
  The expression can specify the charge as a function of voltage, or
  the capacitance as a function of time.
  
\item A {\it model}, which calculates the capacitance as a function of
  length and width, referencing a {\tt .model} statement of type {\tt
    C}.  This is compatible with the Spice-3 ``semiconductor
  capacitor''.

\end{enumerate}
%------------------------------------------------------------------------
\subsection{Model statement}

A model statement may be used,, with model type {\tt C} or {\tt Cap}.
The parameters are:

\begin{description}
  
\item[{\tt CJ} = {\it x}] Junction bottom capacitance. (Farads / meter
  squared).  (Default = 0.)
  
\item[{\tt CJSW} = {\it x}] Junction sidewall capacitance. (Farads /
  meter).  (Default = 0.)
  
\item[{\tt DEFW} = {\it x}] Default width. (meters).  (Default = 1e-6)
  
\item[{\tt NARROW} = {\it x}] Narrowing due to side etching. (meters).
  (Default = 0.)
  
\item[{\tt TC1} = {\it x}] First order temperature coefficient.
  (Farads / degree C).  (Default = 0.) (Not in Spice.)
  
\item[{\tt TC2} = {\it x}] Second order temperature coefficient.
  (Farads / degree C squared).  (Default = 0.) (Not in Spice.)
  
\item[{\tt TNOM} = {\it x}] Parameter measurement temperature.
  (degrees C.).  (Default = 27.) (Not in Spice.)

\end{description}

Capacitance is computed by the formula:

\begin{verbatim}
capacitance = CJ * (L - NARROW) * (W - NARROW)
  + 2 * CJSW * (L + L - 2 * NARROW)
\end{verbatim}

After the nominal value is calculated, it is adjusted for temperature
by the formula:

\begin{verbatim}
value *= (1 + TC1 * (T-T0) + TC2 * (T-T0)^2)
\end{verbatim}
%------------------------------------------------------------------------
%------------------------------------------------------------------------
